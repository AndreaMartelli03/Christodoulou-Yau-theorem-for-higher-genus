\documentclass{article}

%----- Math ---------------------------------------------
\usepackage{amsmath}   
\usepackage{mathrsfs}      
\usepackage{mathtools}       
\usepackage{amssymb}   
\usepackage{amsthm}   
\usepackage{esint}     
\usepackage{resmes} 
\usepackage{stackengine}
\usepackage{amsfonts}
\usepackage{stmaryrd}
\usepackage{dsfont}



%----- Design ------------------------------------------- 
\usepackage{lastpage}                              
\usepackage{enumitem}
\usepackage{multirow}


%----- Intestazioni ---------------------------------------
\usepackage{fancyhdr}
\pagestyle{fancy}
\fancyhf{} % Clear all headers and footers
\fancyhead[R]{\nouppercase{\leftmark}} % Header right on all pages
\fancyfoot[C]{\thepage} % Footer center, page number
\setlength{\headheight}{14.5pt}

% Pacchetto titlesec per personalizzare i titoli dei capitoli
\usepackage{titlesec}

% Comando per settare l'intestazione correttamente per l'introduzione
\newcommand{\chapterwithintroduction}[1]{
	\chapter*{#1}
	\addcontentsline{toc}{chapter}{#1}
	\markboth{#1}{}
}

%----- Pacchetti Disegno --------------------------------
\usepackage{tikz}
\makeatother 
\usetikzlibrary{3d,perspective}
\usetikzlibrary{patterns}
\usetikzlibrary{arrows,calc,patterns}
\usepackage{tikz-cd}
\usepackage{graphicx}
\usepackage{thmtools}
\usepackage{xcolor}
\usepackage[all]{xy}        
\usetikzlibrary{decorations.markings}
\usetikzlibrary{hobby}
\usepackage{pgfplots}
\pgfplotsset{compat=1.18}


%----- Symbols -----------------------------------------
\input{symbols}

%----- Hyperref ------------------------------------------
\usepackage{nameref}
\usepackage{csquotes}
\usepackage{hyperref}
\hypersetup{
	colorlinks=true,
	linkcolor=black, % Color for normal internal links
	filecolor=black, % Color for file links
	urlcolor=blue, % Color for external links
	citecolor=blue % Color for citations
}
\usepackage{cleveref} 


% -----Ambienti matematici-------------------------------
\theoremstyle{plain}
\newtheorem{thm}{Theorem}[section]
\newtheorem*{thm*}{Theorem}
\newtheorem{prop}[thm]{Proposition}
\newtheorem{cor}[thm]{Corollary}
\newtheorem{lemma}[thm]{Lemma}
\newtheorem{claim}{Claim}
\theoremstyle{definition}
\newtheorem{defi}[thm]{Definition}
\newtheorem{axiom}{Axiom}
\theoremstyle{remark}
\newtheorem{rmk}[thm]{Remark}
\newtheorem{ex}[thm]{Example}
\newtheorem{exercise}{Exercise}


%------BIBLIOGRAFIA-----------
\usepackage[style=alphabetic, backend=biber]{biblatex}
% Definizione di un nuovo driver per nascondere URL e DOI
\AtEveryBibitem{
 	\clearfield{url} % Nasconde l'URL
 	\clearfield{doi} % Nasconde il DOI
 	\clearfield{isbn} % Nasconde ISBN
 	\clearfield{issn} % Nasconde ISSN
 	\clearfield{note} % Nasconde le note
}
\addbibresource{biblio.bib}
\title{Christodoulou-Yau theorem for higher genus}
\author{Andrea Martelli}
\date{May 2025}

\begin{document}

\maketitle

\section{Introduction}

It holds the following theorem, proved by D. Christodoulou and S.T. Yau:
\begin{thm}[Christodoulou, Yau]\label{thm: Christodoulou-Yau}
    Let \((M^3,g)\) be a Riemannian 3-manifold and let \(\Sigma^2 \subset (M^3,g)\) be volume-preserving stable CMC sphere. Then
    \begin{equation}\label{eq: CY eq}
        16\pi - H^2|\Sigma| \ge \frac{2}{3}\int_\Sigma (R+|h^\circ|^2) \ \dif \sigma.
    \end{equation}
    In particular, if \((M,g)\) has positive scalar curvature \(R\ge 0\), the Hawking mass of \(\Sigma\) is non-negative:
    \[
        m_{\mathrm{Haw}}(\Sigma) \coloneq \sqrt{\frac{|\Sigma|}{16\pi}}\left( 16\pi -\int_\Sigma H^2 \dif \sigma \right) \ge 0.
    \]
\end{thm}
This result is important because CMC stable spheres often appears in General Relativity as boundary of black holes, as in the Schwarzschild space. Then one can use the Hawking mass to define a total mass of the space. Using this theorem, if the ambient manifold \(M\) has non negative scalar curvature, the total mass will be non negative. 

However, the proof can be generalized to orientable surfaces \(\Sigma\) of any genus. The trick is to borrow the Brill-Noether theorem from algebraic geometry to construct a conformal branched map to the sphere \(\Sp^2\) with degree bound from above by the genus. Remarkably, at the end the the Gauss-Bonnet term cancels out with this upper bound, giving essentially the same result of the sphere:
\begin{thm}\label{thm: Christodoulou-Yau for higher genus}
    Let \((M^3,g)\) be a Riemannian 3-manifold and let \(\Sigma^2 \subset (M^3,g)\) be volume-preserving stable CMC surface of genus \(k\). Then
    \begin{itemize}
        \item if \(k\) is even,
        \[
        16\pi - H^2|\Sigma| \ge \frac{2}{3}\int_\Sigma (R+|h^\circ|^2) \ \dif \sigma;
        \]  
        \item if \(k\) is odd,
        \[
        \frac{64\pi}{3} - H^2|\Sigma| \ge \frac{2}{3}\int_\Sigma (R+|h^\circ|^2) \ \dif \sigma.
        \]
    \end{itemize}
\end{thm}

This is not exactly what is expected. Indeed one could expect to see exactly the critical bounds for the Willmore inequality. In 1965 Willmore showed that round spheres are the only compact surfaces in \(\R^3\) that minimize the Willmore energy
\[
    W(\Sigma) = \int_\Sigma H^2 \dif \sigma,
\]
that is
\[
    \int_\Sigma H^2 \dif \sigma \ge 16\pi
\]
with equality if and only if \(\Sigma\) is a round sphere in \(\R^3\). This is perfectly aligned with Christodoulou-Yau inequality, as in \(\R^3\) the only CMC spheres are the round spheres and thus the left hand side of \eqref{eq: CY eq} is 0. 

Let us try to minimize the Willmore energy among compact surfaces of genus at least \(1\). In 2014, Marques and Neves \cite{MarquesNeves_Willmore} proved that for every surface \(\Sigma \subset \R^3\) with genus at least 1,
\[
    \int_\Sigma H^2 \ge 8\pi^2
\]
with equality if and only if \(\Sigma\) is a torus obtained by stereographic projecting the Clifford torus \(\S^1(1/\sqrt{2}) \times \S^1(1/\sqrt{2})\) of the round 3-sphere \(\Sp^3\). However, even if we fix the genus to be 1, in Theorem~\ref{thm: Christodoulou-Yau for higher genus} we lost the previous symmetry.
\\

I added a small appendix about the Brill-Noether theorem. Actually, it is not the statement that one can usually find in Algebraic Geometry textbooks, but it is the result of an office hour with professor Roberto Pignatelli. However, I take responsibility for any mistake.
\\

\section{Christodoulou-Yau theorem for higher genus}
The proof of Theorem~\ref{thm: Christodoulou-Yau} uses a balancing trick to choose a good average-free test function. We can make this trick work also for orientable closed surfaces of higher genus.

The first step is to move the problem to a sphere ‘‘up to a multiplicity''. We will exploit this result from the theory of Riemann  surfaces (see Theorem~\ref{thm: Brill-Noether}):
\begin{thm}
	If \(\Sigma\) is a closed Riemann surface of genus \(k\), then there exists a holomorphic map\footnote{If we endow with Riemannian structures (\(\Sp^2\) has the round metric), \(\Phi\) is a branched conformal map, i.e. a conformal map that is a covering of degree \(\deg(\Phi)\) away from a finitely many points.} \(\Phi \colon \Sigma \to \C\Pbb^1 \cong \Sp^2\) with degree at most
	\begin{equation}\label{eq: Riemann-Roch bounds on the degree}
		\deg(\Phi) \leq \begin{cases}
			1+k/2 & \text{if }k \text{ is even}\\
			3/2+ k/2 &\text{if }k \text{ is odd}.
		\end{cases}
	\end{equation}
\end{thm}

This means that
\[
	\Phi^* g_0 = \mu^2g
\]
for some function \( \mu \colon \Sigma \to \R\). Then we have natural maps
\[
\begin{tikzcd}
	\Sigma \ar[dr, dashed] \arrow[r, "\Phi"] &\Sp^2 \ar[d, "x^i"]\\
	& \R
\end{tikzcd}
\]
where \(x^i \colon \Sp^2 \subset \R^3 \to \R\) is the \(i\)-th coordinate function of \(\R^3\). In particular, 
\begin{equation}\label{eq: sum (xi.Phi)^2=1}
	\sum_{i=1}^3 (x^i \circ \Phi)^2=1
\end{equation}
at every point. 

For simplicity, suppose that \(\Sigma \cong \Sp^2\) is a sphere, so that actually we can choose \(\Phi\) as a diffeomorphism with the standard sphere of \(\R^3\), which is conformal because all 2-surfaces are locally conformal, and has \(\deg(\Phi)=1\).
\begin{lemma}\label{lem: energy and conformal diffeo}
	Let \((\Sigma^2,g)\) be a Riemannian 2-manifold. The energy 
    \[
        E_g(u) = \int_\Sigma |\nabla_g u |_g^2 \ \dif \sigma_g
    \]
    of a function \(u \in C^1(\Sigma)\) does not change under conformal diffeomorphisms. In other words, if \(\phi \colon (\widetilde{\Sigma},\widetilde{g})\to (\Sigma,g)\) is a conformal diffeomorphism,
    \[
        E_g(u) = E_{\widetilde{g}}(u \circ \phi).
    \]
\end{lemma}
\begin{proof}
	Observe that
    \[
        E_{\widetilde{g}}(u \circ \phi) = E_{\phi^* \widetilde{g}} (u),
    \]
    so we can actually assume that \(\widetilde{\Sigma}=\Sigma\) and \(\widetilde{g}=\lambda g\) for some positive smooth function \(\lambda \colon \Sigma \to (0,+\infty)\). On some coordinates \(x^i\)
    \begin{align*}
        \dif \sigma_{\widetilde{g}} &= \sqrt{\det(\widetilde{g}_{ij})} \dif x^1 \wedge \dif x^2\\
        &= \sqrt{\det(\lambda{g}_{ij})} \dif x^1 \wedge \dif x^2\\
        &= \lambda \sqrt{\det({g}_{ij})} \dif x^1 \wedge \dif x^2 = \lambda \dif \sigma_g.
    \end{align*}
    On the other hand,
    \[
        |\nabla_g u|_g^2 = g_{ij}(\nabla_g u)^i (\nabla_g u)^j = g_{ij}g^{ik}g^{j\ell}\frac{\de u}{\de x^k}\frac{\de u}{\de x^\ell},
    \]
    thus
    \[
        |\nabla_{\widetilde{g}} u|^2_{\widetilde{g}} = \frac{1}{\lambda} |\nabla_g u|_g^2.
    \]
    Putting everything together,
    \[
        E_{\phi^* \widetilde{g}} (u) = \int_\Sigma \frac{1}{\lambda} |\nabla_g u|_g^2 \lambda \dif \sigma_g = E_g(u). \qedhere
    \]
\end{proof}
Therefore we have:
\begin{equation*}
	\int_\Sigma |\nabla_g (x^i \circ \Phi)|_g^2 = \int_{\Sp^2} |\nabla_{g_0} x^i |_{g_0}^2.
\end{equation*}
In the general case, consider the smooth 2-form
\[
    \omega = |\nabla_{g_0} x^i|_{g_0}^2 \dif \sigma_{g_0} \in \Omega^2(\Sp^2)
\]
where \(\dif \sigma_{g_0}\) is the volume element of the round sphere \((\Sp^2,g_0)\). Then observe that\footnote{
The \emph{Riemannian volume element} \(\dif \mu\) of an orientable Riemannian manifold \((M,g)\) is defined as the unique \(n\)-form \(\dif \mu\) (the notation mimics the notation for integral with respect to a measure, but there is no form \(\mu\) and \(\dif \mu\) is not a differential of anything) such that 
\[
    (\dif \mu)_p(E_1,\dots,E_n)=1
\]
for any positively oriented orthonormal basis \(E_1,\dots,E_n\) of \((T_pM,g_p)\).
}
\[
    \Phi^*(\dif \sigma_{g_0}) = \dif \sigma_{\Phi^*g_0}
\]
and 
\[
    |(\nabla_{g_0} x^i)\circ \Phi|_{g_0}^2 = |\nabla_{\Phi^*g_0} (x^i \circ \Phi)|_{\Phi^* g_0}^2
\]
essentially by definition of pullback metric, on every point in which \(\Phi\) is a local diffeomorphism (at every point except at branch points).
Therefore
\[
    \Phi^* \omega = |\nabla_{\Phi^*g_0} (x^i \circ \Phi)|_{\Phi^* g_0}^2 \dif \sigma_{\Phi^*g_0}
\]
and since \(g\) and \(\Phi^*g_0\) are conformal metrics, by Lemma~\ref{lem: energy and conformal diffeo}
\begin{align}\label{eq: Hersch - conformal invariance}
    \int_\Sigma |\nabla_g (x^i \circ \Phi)|_g^2 \ \dif \sigma_g &= \int_\Sigma \nabla_{\Phi^*g_0} (x^i \circ \Phi)|_{\Phi^* g_0}^2 \dif \sigma_{\Phi^*g_0} \nonumber\\
    &= \int_\Sigma \Phi^* \omega =\deg(\Phi) \int_\Sigma \omega \nonumber\\
    &= \deg(\Phi) \int_\Sigma |\nabla_{g_0} x^i|_{g_0}^2 \dif \sigma_{g_0}.
\end{align}

Moreover, 
\[
	\nabla_{g_0} x^i = (\nabla^{\R^3} x^i)^\top = e_i^\top
\]
so
\[
	\sum_{i=1}^3 |\nabla_{g_0} x^i|^2 = \sum_{i=1}^3 (1-(x\cdot e_i)^2) = 3- |x|^2 = 2.
\]
Then, putting together this computation with \eqref{eq: Hersch - conformal invariance}, we get
\begin{equation}\label{eq: energy computation}
	\sum_{i=1}^3 \int_\Sigma |\nabla_\Sigma (x^i \circ \Phi)|^2 = \deg(\Phi) \sum_{i=1}^3\int_{\Sp^2}|\nabla_{\Sp^2} x^i|^2 = 8\pi \deg(\Phi).
\end{equation}

Now we perform the so called Hersch's balancing trick. Given any \(\vp_1 \in L^1(\Sigma)\), possibly post-composing with a conformal diffeomorphism of \((\Sp^2,g_0)\), i.e. replacing \(\Phi\) with \(\gamma \circ \Phi\) for some \(\gamma \in \mathrm{Conf}(\Sp^2,g_0)\), we can achieve that
\begin{equation*}
	\int_{\Sigma}(x^i \circ \Phi) \vp_1 = 0 \qquad i=1,2,3.
\end{equation*}

Let us sketch a proof of this fact. The group \(\mathrm{Conf}(\Sp^2,g_0)\) of conformal diffeomorphisms of the round sphere contains \(B^3_1(0) = B^3\) as a subgroup as follows: \(0\) is the identity, while any other point \(a = ru \in B^3\), with \(r \in [0,1)\) and \(u \in \Sp^2\), is identified with the dilation of \(\Sp^2\) fixing the opposite poles \(u\) and \(-u\) and dilating out from \(-u\) by a factor \((1-r)^{-1}\), namely
\[
	x \mapsto \begin{cases}
		\pi_{-u}^{-1}\left(\frac{1}{1-r}\pi_{-u}(x) \right) &x\neq -u \\
		-u & x= -u
	\end{cases}
\]
for each \(x \in \Sp^2\), where \(\pi_v \colon \Sp^2 \to \C\) is the stereographic projection from \(v\in \Sp^2\). Moreover, note that as \(\gamma \to u \in \de B^3\)
\begin{equation}\label{eq: Hersch - key prop for extension to boundary}
	\gamma(x) \to u
\end{equation}
for each \(x \in \Sp^2\).
Now consider the map
\begin{align*}
	\Psi \colon B^3 \subset \mathrm{Conf}(\Sp^2,g_0) &\to \R^3 
\end{align*}
defined by 
\begin{equation*}
	\Psi(\gamma) \coloneq \frac{1}{\int_\Sigma \vp_1} \int_\Sigma (\gamma \circ \Phi)\vp_1 = \frac{1}{\int_\Sigma \vp_1 }\left( \int_\Sigma (x^i \circ \gamma \circ \Phi)\vp_1 \right)_{i=1,2,3}
\end{equation*}
for any \(\gamma \in B^3\). Clearly \(\Psi\) is a continuous function.
We need to show that there exists \(\gamma_0 \in B^3\) such that \(\Psi(\gamma_0)=0\). By triangular inequality, \(\Psi(B^3) \subset \overline{B^3}\):
\begin{equation*}
	\frac{1}{\int_\Sigma \vp_1} \left| \int_\Sigma (\gamma \circ \Phi)\vp_1 \right| \le \frac{1}{\int_\Sigma \vp_1} \int_\Sigma |\gamma \circ \Phi| \vp_1 = 1
\end{equation*}
By \eqref{eq: Hersch - key prop for extension to boundary}, as \(\gamma \to u \in \de B^3\)
\[
	\int_\Sigma (\gamma \circ \Phi)\vp_1 \to \left(\int_\Sigma \vp_1\right) u,
\]
so there exists a unique continuous extension \(\Psi \in C^0(\overline{B^3}, \overline{B^3})\) such that \(\Psi|_{\de B^3} = \Id_{B^3}\). We claim that \(\Psi\) is surjective, and in particular exists \(\gamma_0 \in G\) such that \(\Psi(\gamma_0)=0\). Indeed, we can argue as in the classical proof of Brouwer fixed point theorem. We know that there is no retraction of \(\overline{B^3}\) on its boundary \(\de B^3\) (for example, because the homology group \(H_2(\de B^3) \cong \Z\) cannot be isomorphic to any subgroup of \(H_2(\overline{B^3})=0\)). Suppose by contradiction that there exists \(x_0 \in B^3 \setminus \Psi(\overline{B^3})\). Then we can define a retraction \(r \colon \overline{B^3} \to \de B^3 \) of \(\overline{B^3}\) on its boundary by letting \(r(x)\) be the intersection point between \(\de B^3\) and the half line starting from \(x_0\) and passing through \(\Psi(x)\):
	\[
		r(x) = x_0 + \lambda(x)(x-x_0),
	\]
where \(\lambda(x) \in (0,1]\) is such that \(|r(x)|=1\). This is a contradiction, so \(\Psi\) is surjective.

Basically, we have proved the following.
\begin{lemma}[Hersch's balancing trick]\label{lem: Hersch trick}
	Let \(\Sigma^2 \subset (M^3,g)\) be a closed Riemann surface and \(\vp \in L^1(\Sigma)\) a function with non zero average. Then there always exists a branched conformal map \(\Phi \colon (\Sigma,g) \to (\Sp^2,g_0)\) having degree bounded by the genus of \(\Sigma\) as in \eqref{eq: Riemann-Roch bounds on the degree} and such that 
	\begin{equation}\label{eq: x_i orthogonal to phi_1}
		\int_\Sigma (x^i \circ \Phi) \vp = 0
	\end{equation}
	for each \(i=1,2,3\), where \(x^i\) is the restriction to \(\Sp^2 \subset \R^3\) of the \(i\)-th coordinate of \(\R^3\).
\end{lemma}


\begin{proof}[Proof of Theorem\ref{thm: Christodoulou-Yau for higher genus}]
    Take \(\Phi \colon \Sigma \to \Sp^2\) as in Hersch's trick~\ref{lem: Hersch trick}. Recall that it holds
    \[
        \sum_{i=1}^3 \int_\Sigma |\nabla_\Sigma (x^i \circ \Phi)|^2 = 8 \pi \deg(\Phi).
    \]
    Using the volume-preserving stability inequality
    \[
        \int_\Sigma |\nabla_\Sigma (x^i \circ \Phi)|^2 \ge \int_\Sigma \left[ |h|^2 + \Ric^M(\nu,\nu)\right]|(x^i \circ \Phi)^2.
    \]
    Now we perform the so called ‘‘Schoen-Yau rearrangement trick'': using the twice contracted Gauss equation,
    \begin{align*}
        |h|^2 + \Ric(\nu,\nu) &= |h|^2 + \frac{1}{2}[R -R^\Sigma + H^2-|h|^2]\\
        &= \frac{1}{2}|h|^2 + \frac{1}{2}R -\frac{1}{2}R^\Sigma + \frac{1}{2}H^2 \\
        &=\frac{1}{2}\left(|h^\circ|^2+R\right) + \frac{3}{4}H^2 -\frac{1}{2}R^\Sigma.
    \end{align*}
    Substituting
    \[
        \int_\Sigma |\nabla_\Sigma (x^i \circ \Phi)|^2 \ge \int_\Sigma \left[ \frac{(|h^\circ|^2+R)}{2} + \frac{3}{4}H^2 -\frac{R^\Sigma}{2}\right]|(x^i \circ \Phi)^2,
    \]
    and summing over \(i\),
    \[
        8\pi \deg(\Phi) \ge \int_\Sigma  \frac{(|h^\circ|^2+R)}{2} + \frac{3}{4}H^2 -\frac{R^\Sigma}{2}.
    \]
    Recall that by Gauss-Bonnet theorem
    \[
        \int_\Sigma \frac{R^\Sigma}{2} = 2\pi \chi(\Sigma) = 2\pi(2-2k)= 4\pi-4k\pi.
    \]
    Thus 
    \begin{align*}
        \frac{1}{2}\int_\Sigma (|h^\circ|^2+R) \le 8\pi \deg(\Phi) +4\pi-4k\pi - \frac{3}{4}H^2|\Sigma|. 
    \end{align*}
    Assume that \(k\) is even. Then \(\deg(\Phi) \le 1+ k/2\), so
    \begin{align*}
        \frac{1}{2}\int_\Sigma (|h^\circ|^2+R) &\le 8\pi \deg(\Phi) +4\pi-4k\pi - \frac{3}{4}H^2|\Sigma|\\
        &\le 8\pi +4k\pi +4\pi -4k\pi -\frac{3}{4} H^2|\Sigma|\\
        &= 12\pi -\frac{3}{4}H^2|\Sigma|.
    \end{align*}
    Multiplying everything by \(4/3\) we get the result.

    Analogously, if \(k\) is odd we have the upper bound \(\deg(\Phi) \le 3/2 + k/2\), so
    \begin{align*}
        \frac{1}{2}\int_\Sigma (|h^\circ|^2+R) &\le 8\pi \deg(\Phi) +4\pi-4k\pi - \frac{3}{4}H^2|\Sigma|\\
        &\le 16\pi -\frac{3}{4}H^2|\Sigma|.
    \end{align*}
    Then again multiply by \(4/3\).
\end{proof}

\appendix
\section{The Brill-Noether theorem}
A Riemann surface is a compact complex manifold of complex dimension 1, namely a complex curve. They correspond to all orientable compact surfaces: the sphere \(\Sp^2 = \C\Pbb^1\), the torus \(\T^2\) and the tori with multiple holes. The number of holes \(g\) is called \emph{genus}.

It holds the following deep theorem.
\begin{thm}[Brill-Noether]\label{thm: Brill-Noether}
    Let \(C\) be a complex curve with genus \(g\). Then there exists a non degenerate map \(\Phi \colon C \to \C\Pbb^r\) of degree \(d\) if and only if
    \[
        \rho(d,r,g) = g-(r+1)(g-d+r) \ge 0.
    \]
\end{thm}
\begin{rmk}
    This is not the usual formulation of the Brill-Noether theorem. Indeed, \(\rho(d,r,g)\) is a lower bound for the dimension (as a subscheme) of a particular subset of the Picard variety \(\mathrm{Pic}(C)\) of \(C\). When this set is non empty, we can construct such a map \(\Phi\).
\end{rmk}
In the special case where \(r=1\), we get that we can find a map \(\Phi\) of degree \(d\) if and only if
\[
    0 \le \rho(d,1,g) = g-2(g-d+1) = 2d - (g+2),
\]
that is
\[
    d \ge 1 + \frac{g}{2}.
\]

Taking the \(d\) as the least possible value, we get
\[
d = \begin{cases}
    1 + g/2 &\text{if }g\text{ is even},\\
    3/2 + g/2 &\text{if }g\text{ is odd}.
\end{cases}
\]


\printbibliography
\end{document}
